\section{Завдання}
Для наведених прикладів ланцюжків побудувати правила граматики.
Перевірити правильність складання правил за допомогою виведення.
Перевірити наявність непродуктивних та недосяжних символів:
\begin{enumerate}
    \item \verb|m = (a % b) / (a / c) - (a / c) - (c % b) - (a % c);|
    \item \verb|l = (c % b) + (b / c) * (a % b) - (a / c);|
    \item \verb|m = (a % c);|
\end{enumerate}


\section{Опис граматики}
\begin{enumerate}
    \item \verb|I| $\to$ \verb|V = ER;|
    \item \verb|V| $\to$ \verb|a | b | ... | z|
    \item \verb|R| $\to$ \verb|OER | \$|
    \item \verb|O| $\to$ \verb|+ | - | * | / | \%|
    \item \verb|E| $\to$ \verb|(VOV)|
\end{enumerate}


\section{Перевірка граматики}
Приклад для перевірки: \verb|l = (a % b) + (b / c)|

\begin{itemize}
    \item[]  \verb|I|
             \xrightarrow{1   } \verb|V = ER;|
    \item[]  \xrightarrow{2.12} \verb|l = ER;|
    \item[]  \xrightarrow{5   } \verb|l = (VOV)R;|
    \item[]  \xrightarrow{2.1 } \verb|l = (aOV)R;|
    \item[]  \xrightarrow{4.5 } \verb|l = (a\%V)R;|
    \item[]  \xrightarrow{2.2 } \verb|l = (a\%b)R;|
    \item[]  \xrightarrow{3.1 } \verb|l = (a\%b)OER;|
    \item[]  \xrightarrow{4.1 } \verb|l = (a\%b)+ER;|
    \item[]  \xrightarrow{5   } \verb|l = (a\%b)+(VOV)R;|
    \item[]  \xrightarrow{2.2 } \verb|l = (a\%b)+(aOV)R;|
    \item[]  \xrightarrow{4.4 } \verb|l = (a\%b)+(b/V)R;|
    \item[]  \xrightarrow{2.3 } \verb|l = (a\%b)+(b/c)R;|
    \item[]  \xrightarrow{3.2 } \verb|l = (a\%b)+(b/c);|
\end{itemize}


\newpage
\subsection{Перевірка на непродуктивність}
\begin{enumerate}
    \item  T N P X
    \item  T N P X E R I
    \item  нема непродуктивних символів
\end{enumerate}

\subsection{Перевірка на недосяжність}
\begin{enumerate}
    \item  I
    \item  I T N E R
    \item  I T N E R P X
    \item  нема недосяжних символів
\end{enumerate}
