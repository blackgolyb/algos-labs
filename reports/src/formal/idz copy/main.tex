\section{Завдання}
\noindent
1. Для наведених прикладів ланцюжків побудувати правила граматики.
Перевірити правильність складання правил за допомогою виведення.
Перевірити наявність непродуктивних та недосяжних символів:

\begin{enumerate}
    \item \verb|struct t { int a; float c,a,c; char c;};|
    \item \verb|struct l { int c,a; int b; };|
    \item \verb|struct k { float c; char c; int a; };|
\end{enumerate}

\noindent
2. Визначити візуально якого класу належить побудована граматика:
\begin{itemize}
    \item Проста
    \item Слабкорозділена
    \item LL - граматика
\end{itemize}

\noindent
3. Якщо граматики не є простою, то побудувати функції ПЕРШ(), СЛІД()
і множину ВИБІР. Для заданої граматики збудувати команди розпізнавача.

\noindent
4. Роботу розпізнавача перевірити на прикладі розпізнавання ланцюжка.


\newpage
\section{Побудова правил граматики}
\begin{enumerate}
    \item  \verb|I| $\to$ \verb|struct N { ER };|
    \item  \verb|N| $\to$ \verb|a|
    \item  \verb|N| $\to$ \verb|b|
    \item  \verb|N| $\to$ \verb|c|
    \item  \verb|N| $\to$ \verb|t|
    \item  \verb|N| $\to$ \verb|l|
    \item  \verb|N| $\to$ \verb|k|
    \item  \verb|T| $\to$ \verb|int|
    \item  \verb|T| $\to$ \verb|float|
    \item  \verb|T| $\to$ \verb|char|
    \item  \verb|E| $\to$ \verb|TNP;|
    \item  \verb|R| $\to$ \verb|ER|
    \item  \verb|R| $\to$ \verb|$|
    \item  \verb|P| $\to$ \verb|,NP|
    \item  \verb|P| $\to$ \verb|$|
\end{enumerate}


\newpage
\section{Перевірка правил граматики}
Приклад для перевірки: \verb|struct k { int a,b; char c; };|
\begin{itemize}
    \item[]  I $\xrightarrow{1}$ \verb|struct N { ER };|
    \item[]  $\xrightarrow{7}$   \verb|struct k { ER };|
    \item[]  $\xrightarrow{11}$  \verb|struct k { TNP;R };|
    \item[]  $\xrightarrow{8}$   \verb|struct k { int NP;R };|
    \item[]  $\xrightarrow{2}$   \verb|struct k { int aP;R };|
    \item[]  $\xrightarrow{14}$  \verb|struct k { int a,NP;R };|
    \item[]  $\xrightarrow{3}$   \verb|struct k { int a,bP;R };|
    \item[]  $\xrightarrow{15}$  \verb|struct k { int a,b;R };|
    \item[]  $\xrightarrow{12}$  \verb|struct k { int a,b; ER };|
    \item[]  $\xrightarrow{11}$  \verb|struct k { int a,b; TNP;R };|
    \item[]  $\xrightarrow{10}$  \verb|struct k { int a,b; char NP;R };|
    \item[]  $\xrightarrow{4}$   \verb|struct k { int a,b; char cP;R };|
    \item[]  $\xrightarrow{15}$  \verb|struct k { int a,b; char c;R };|
    \item[]  $\xrightarrow{13}$  \verb|struct k { int a,b; char c; };|
\end{itemize}

\newpage
\section{Перевірка на непродуктивність}
\begin{enumerate}
    \item[]  N T R P
    \item[]  N T R P E I
    \item[]  нема непродуктивних символів
\end{enumerate}

\section{Перевірка на недосяжність}
\begin{enumerate}
    \item[]  I
    \item[]  I N E R
    \item[]  I N E R T P
    \item[]  нема недосяжних символів
\end{enumerate}


\newpage
\section{Побудова функції ПЕРШ}
\begin{itemize}
    \item[]  ПЕРШ(\verb|I| $\to$ \verb|struct N { ER };|) = \{\verb|struct|\}
    \item[]  ПЕРШ(\verb|N| $\to$ \verb|a|)                = \{\verb|a|\}
    \item[]  ПЕРШ(\verb|N| $\to$ \verb|b|)                = \{\verb|b|\}
    \item[]  ПЕРШ(\verb|N| $\to$ \verb|c|)                = \{\verb|c|\}
    \item[]  ПЕРШ(\verb|N| $\to$ \verb|t|)                = \{\verb|t|\}
    \item[]  ПЕРШ(\verb|N| $\to$ \verb|l|)                = \{\verb|l|\}
    \item[]  ПЕРШ(\verb|N| $\to$ \verb|k|)                = \{\verb|k|\}
    \item[]  ПЕРШ(\verb|T| $\to$ \verb|int|)              = \{\verb|int|\}
    \item[]  ПЕРШ(\verb|T| $\to$ \verb|float|)            = \{\verb|float|\}
    \item[]  ПЕРШ(\verb|T| $\to$ \verb|char|)             = \{\verb|char|\}
    \item[]  ПЕРШ(\verb|E| $\to$ \verb|TNP;|)             = ПЕРШ(\verb|T|) = \{\verb|int|, \verb|float|, \verb|char|\}
    \item[]  ПЕРШ(\verb|R| $\to$ \verb|ER|)               = ПЕРШ(\verb|E|) = ПЕРШ(\verb|T|) = \{\verb|int|, \verb|float|, \verb|char|\}
    \item[]  ПЕРШ(\verb|R| $\to$ \verb|$|)                = \{\verb|$|\}
    \item[]  ПЕРШ(\verb|P| $\to$ \verb|,NP|)              = \{\verb|,|\}
    \item[]  ПЕРШ(\verb|P| $\to$ \verb|$|)                = \{\verb|$|\}
\end{itemize}

\section{Побудова функції СЛІД}
\begin{itemize}
    \item[]  СЛІД(\verb|I|) = \{\verb|$|\}
    \item[]  СЛІД(\verb|N|) = \{\verb|{|\} $\cup$ ПЕРШ(\verb|P|) = \{\verb|{|, \verb|,|, \verb|;|\}
    \item[]  СЛІД(\verb|T|) = ПЕРШ(\verb|N|) = \{\verb|a|, \verb|b|, \verb|c|, \verb|t|, \verb|l|, \verb|k|\}
    \item[]  СЛІД(\verb|E|) = ПЕРШ(\verb|R|) $\cup$ \{\verb|}|\} = \{\verb|int|, \verb|float|, \verb|char|, \verb|}|\}
    \item[]  СЛІД(\verb|R|) = \{\verb|}|\} $\cup$ СЛІД(\verb|R|) = \{\verb|}|\}
    \item[]  СЛІД(\verb|P|) = \{\verb|;|\} $\cup$ СЛІД(\verb|P|) = \{\verb|;|\}
\end{itemize}

\section{Побудова функції ВИБІР}
\begin{itemize}
    \item[]  ВИБІР(\verb|I| $\to$ \verb|struct N { ER };|) = ПЕРШ(\verb|1|)  = \{\verb|struct|\}
    \item[]  ВИБІР(\verb|N| $\to$ \verb|a|)                = ПЕРШ(\verb|2|)  = \{\verb|a|\}
    \item[]  ВИБІР(\verb|N| $\to$ \verb|b|)                = ПЕРШ(\verb|3|)  = \{\verb|b|\}
    \item[]  ВИБІР(\verb|N| $\to$ \verb|c|)                = ПЕРШ(\verb|4|)  = \{\verb|c|\}
    \item[]  ВИБІР(\verb|N| $\to$ \verb|t|)                = ПЕРШ(\verb|5|)  = \{\verb|t|\}
    \item[]  ВИБІР(\verb|N| $\to$ \verb|l|)                = ПЕРШ(\verb|6|)  = \{\verb|l|\}
    \item[]  ВИБІР(\verb|N| $\to$ \verb|k|)                = ПЕРШ(\verb|7|)  = \{\verb|k|\}
    \item[]  ВИБІР(\verb|T| $\to$ \verb|int|)              = ПЕРШ(\verb|8|)  = \{\verb|int|\}
    \item[]  ВИБІР(\verb|T| $\to$ \verb|float|)            = ПЕРШ(\verb|9|)  = \{\verb|float|\}
    \item[]  ВИБІР(\verb|T| $\to$ \verb|char|)             = ПЕРШ(\verb|10|) = \{\verb|char|\}
    \item[]  ВИБІР(\verb|E| $\to$ \verb|TNP;|)             = ПЕРШ(\verb|11|) = \{\verb|int|, \verb|float|, \verb|char|\}
    \item[]  ВИБІР(\verb|R| $\to$ \verb|ER|)               = ПЕРШ(\verb|12|) = \{\verb|int|, \verb|float|, \verb|char|\}
    \item[]  ВИБІР(\verb|R| $\to$ \verb|$|)                = СЛІД(\verb|R|)  = \{\verb|}|\}
    \item[]  ВИБІР(\verb|P| $\to$ \verb|,NP|)              = ПЕРШ(\verb|14|) = \{\verb|,|\}
    \item[]  ВИБІР(\verb|P| $\to$ \verb|$|)                = СЛІД(\verb|P|)  = \{\verb|;|\}
\end{itemize}


\section{Визначення типу граматики}
Граматика є LL(1) граматикою


\newpage
\section{Побудова команд розпізнавача}
\begin{enumerate}
    \item  $f$  (\verb|s|,  \verb|struct|,  \verb|I|)       = \verb|(s, ;}RE{N)|\
    \item  $f$  (\verb|s|,  \verb|a|,       \verb|N|)       = \verb|(s, $)|\
    \item  $f$  (\verb|s|,  \verb|b|,       \verb|N|)       = \verb|(s, $)|\
    \item  $f$  (\verb|s|,  \verb|c|,       \verb|N|)       = \verb|(s, $)|\
    \item  $f$  (\verb|s|,  \verb|t|,       \verb|N|)       = \verb|(s, $)|\
    \item  $f$  (\verb|s|,  \verb|l|,       \verb|N|)       = \verb|(s, $)|\
    \item  $f$  (\verb|s|,  \verb|k|,       \verb|N|)       = \verb|(s, $)|\
    \item  $f$  (\verb|s|,  \verb|int|,     \verb|T|)       = \verb|(s, $)|\
    \item  $f$  (\verb|s|,  \verb|float|,   \verb|T|)       = \verb|(s, $)|\
    \item  $f$  (\verb|s|,  \verb|char|,    \verb|T|)       = \verb|(s, $)|\
    \item  $f*$ (\verb|s|,  \verb|int|,     \verb|E|)       = \verb|(s, ;PNT)|\
    \item  $f*$ (\verb|s|,  \verb|float|,   \verb|E|)       = \verb|(s, ;PNT)|\
    \item  $f*$ (\verb|s|,  \verb|char|,    \verb|E|)       = \verb|(s, ;PNT)|\
    \item  $f*$ (\verb|s|,  \verb|int|,     \verb|R|)       = \verb|(s, RE)|\
    \item  $f*$ (\verb|s|,  \verb|float|,   \verb|R|)       = \verb|(s, RE)|\
    \item  $f*$ (\verb|s|,  \verb|char|,    \verb|R|)       = \verb|(s, RE)|\
    \item  $f*$ (\verb|s|,  \verb|}|,       \verb|R|)       = \verb|(s, $)|\
    \item  $f$  (\verb|s|,  \verb|,|,       \verb|P|)       = \verb|(s, PN)|\
    \item  $f*$ (\verb|s|,  \verb|;|,       \verb|P|)       = \verb|(s, $)|\

    \item  $f$  (\verb|s|,  \verb|{|,       \verb|{|)       = \verb|(s, $)|\
    \item  $f$  (\verb|s|,  \verb|}|,       \verb|}|)       = \verb|(s, $)|\
    \item  $f$  (\verb|s|,  \verb|;|,       \verb|;|)       = \verb|(s, $)|\
    \item  $f*$ (\verb|s|,  \verb|$|,       \verb|h0|)      = \verb|(s, $)|\

\end{enumerate}

\newcommand{\xvdash}[1]{$\stackrel{\text{#1}}{\vdash}$}

\newpage
\section{Перевірка команд розпізнавача}
Приклад для перевірки: \verb|struct k { int a,b; char c; };|
\begin{itemize}
    \item[]  (s, \quad \verb|struct k { int a,b; char c; };|,    \quad $h_{0}$\verb|I|)            $\vdash$ 1
    \item[]  (s, \quad \verb|k { int a,b; char c; };|,           \quad $h_{0}$\verb|;}RE{N|)       $\vdash$ 7
    \item[]  (s, \quad \verb|{ int a,b; char c; };|,             \quad $h_{0}$\verb|;}RE{|)        $\vdash$ 20
    \item[]  (s, \quad \verb|int a,b; char c; };|,               \quad $h_{0}$\verb|;}RE|)         $\vdash$ 11
    \item[]  (s, \quad \verb|int a,b; char c; };|,               \quad $h_{0}$\verb|;}R;PNT|)      $\vdash$ 8
    \item[]  (s, \quad \verb|a,b; char c; };|,                   \quad $h_{0}$\verb|;}R;PN|)       $\vdash$ 2
    \item[]  (s, \quad \verb|,b; char c; };|,                    \quad $h_{0}$\verb|;}R;P|)        $\vdash$ 18
    \item[]  (s, \quad \verb|b; char c; };|,                     \quad $h_{0}$\verb|;}R;PN|)       $\vdash$ 3
    \item[]  (s, \quad \verb|; char c; };|,                      \quad $h_{0}$\verb|;}R;P|)        $\vdash$ 19
    \item[]  (s, \quad \verb|; char c; };|,                      \quad $h_{0}$\verb|;}R;|)         $\vdash$ 22
    \item[]  (s, \quad \verb|char c; };|,                        \quad $h_{0}$\verb|;}R|)          $\vdash$ 16
    \item[]  (s, \quad \verb|char c; };|,                        \quad $h_{0}$\verb|;}RE|)         $\vdash$ 13
    \item[]  (s, \quad \verb|char c; };|,                        \quad $h_{0}$\verb|;}R;PNT|)      $\vdash$ 10
    \item[]  (s, \quad \verb|c; };|,                             \quad $h_{0}$\verb|;}R;PN|)       $\vdash$ 4
    \item[]  (s, \quad \verb|; };|,                              \quad $h_{0}$\verb|;}R;P|)        $\vdash$ 19
    \item[]  (s, \quad \verb|; };|,                              \quad $h_{0}$\verb|;}R;|)         $\vdash$ 22
    \item[]  (s, \quad \verb|};|,                                \quad $h_{0}$\verb|;}R|)          $\vdash$ 17
    \item[]  (s, \quad \verb|};|,                                \quad $h_{0}$\verb|;}|)           $\vdash$ 21
    \item[]  (s, \quad \verb|;|,                                 \quad $h_{0}$\verb|;|)            $\vdash$ 22
    \item[]  (s, \quad \verb|$|,                                 \quad $h_{0}$)                    $\vdash$ 23
    \item[]  (s, \quad \verb|$|,                                 \quad \verb|$|) - Успішне розпізнавання
\end{itemize}
